\newcommand{\true}{\mathsf{true}}
\newcommand{\false}{\mathsf{false}}
\newcommand{\D}{\mathcal{D}}

\newcommand{\nbdd}{\textsf{nBDD}\xspace}
\newcommand{\ubdd}{\textsf{uBDD}\xspace}
\newcommand{\bdd}{\textsf{BDD}\xspace}

\newcommand{\nfbdd}{\textsf{nFBDD}\xspace}
\newcommand{\nobdd}{\textsf{nOBDD}\xspace}

\newcommand{\ndt}{\textsf{nDT}\xspace}

\newcommand{\ufbdd}{\textsf{uFBDD}\xspace}
\newcommand{\uobdd}{\textsf{uOBDD}\xspace}

\newcommand{\fbdd}{\textsf{FBDD}\xspace}
\newcommand{\obdd}{\textsf{OBDD}\xspace}

\newcommand{\udt}{\textsf{uDT}\xspace}
\newcommand{\ddt}{\textsf{DT}\xspace}

\newcommand{\nodt}{\textsf{nODT}\xspace}
\newcommand{\uodt}{\textsf{uODT}\xspace}
\newcommand{\odt}{\textsf{ODT}\xspace}


\todo[inline]{Marcelo \& Weiming}

%Dimensions: structuredness of variables, and degree of ambiguity 
%Paragraphs on decision trees
%zero-suppresed
%complete

A \emph{nondeterministic binary decision
diagram} (\nbdd~\citep{BW97,ACMS20}) over  a
set of variables $\X$ is a rooted\footnote{All nodes have a directed path from the root.} directed
acyclic graph~$\D$ with labels on edges and
nodes such that: (i) each
edge is labeled with either the symbol $0$ or
the symbol $1$; (ii) each leaf (i.e., a node
with no outgoing edges) is labeled
with~$\true$ or $\false$; and (iii) each internal
node (a node that is not a leaf) is labeled
with a variable $X \in \X$ and has at least one outgoing edge labeled by $1$ and at least one outgoing edge labeled by $0$. A run $\pi$ of $\D$ is a path
$u_1 \cdots u_k$ of $\D$ such that $u_1$ is
the root of $\D$ and $u_k$ is a leaf of $\D$.
A run $\pi = u_1 \cdots u_k$ of $\D$ is
accepting if the label of $u_k$ is $\true$. An
assignment $\as \in \assign(\X)$ and a run
$\pi = u_1 \cdots u_k$ of $\D$ are said to be
consistent if for every $i \in [1, k-1]$, letting $X$ be the variable that labels $u_i$, it
holds that $\as(X)$ is the label of the edge
from $u_i$ to $u_{i+1}$. Moreover, an
assignment $\as$ is accepted by $\D$, denoted
by $\D(\as) = 1$ or $\as \models \D$, if there exists an accepting
run $\pi$ of $\D$ that is consistent with $\as$; otherwise, the
assignment $\as$ is rejected by $\D$, which is
denoted by $\D(\as) = 0$ (or $\as \not\models \D$). Notice that for every assignment $\as$, by condition (iii) there is always at least one run that is consistent with $\as$. Furthermore, observe that $\D$ represents a Boolean function over variables $\X$, and we will often identify $\D$ with the Boolean function that it represents. 


\begin{example}
  An \nbdd $\D$ over the set of variables $\X = \{X,Y,Z,W\}$ is shown in Figure~\ref{fig-nbdd-with-id}. For each node we include its identifier and label; for instance, $u_1 : X$ indicates that node $u_1$ has label $X$. Two runs of $\D$ are $\pi_1 = u_1,u_2,u_6$ and $\pi_2 = u_1, u_3, u_5, u_7$, where $\pi_1$ is accepting as $u_6$ is a leaf with label $\true$. Consider the assignment $\as_1 \in \assign(\X)$ such that $\as_1(X) = \as_1(Y) = 0$ and $\as_1(Z) = \as_1(W) = 1$. Then we have that $\as_1$ is consistent with $\pi_1$. Notice that an assignment can be consistent with many different runs of an \nbdd; for instance, $\as_1$ is consistent with runs $\pi_1$ and $\pi_2$, as well as with run $u_1, u_3, u_6$. An assignment is accepted by an \nbdd if there exists at least one accepting run of the \nbdd that is consistent with it; for instance $\D(\as_1) = 1$ since $\as_1$ is consistent with the accepting run $\pi_1$ ($\D(\as_1) = 1$ despite the fact that $\as_1$ is consistent with the run $\pi_2 = u_1, u_3, u_5, u_7$ and the label of leaf $u_7$ is $\false$). Also notice that an assignment may not be consistent with any run \comm{Mikaël}{not anymore with the new definitions, update} of an \nbdd, in which case it is rejected by the \nbdd. For instance, if $\as_2$ is the assignment such that $\as_2(X) = 1$ and $\as_2(Y) = \as_2(Z) = \as_2(W) = 0$, then $\D(\as_2) = 0$ since $\as_2$ is not consistent with any run of $\D$.

We depict in Figure \ref{fig-nbdd-without-id} the same \nbdd as in Figure \ref{fig-nbdd-with-id}, but without including node identifiers. The usual graphical representation of an \nbdd is the one given in  Figure \ref{fig-nbdd-without-id}.
\end{example}

\begin{figure}
\begin{center}
\begin{subfigure}{0.45\textwidth}
\begin{tikzpicture}[scale=0.85, transform shape]
  \node[thick, draw=black, ellipse, minimum height=8mm, text centered] (n1) {$u_1 : X$};
  \node[thick, draw=black, ellipse, minimum height=8mm, below=12mm of n1, text centered] (n3) {$u_3 : W$}
    edge[arrin] node[left] {$0$} (n1);
  \node[thick, draw=black, ellipse, minimum height=8mm, left=12mm of n3, text centered] (n2) {$u_2 : Z$}
    edge[arrin] node[above] {$0$} (n1);
  \node[thick, draw=black, ellipse, minimum height=8mm, right=12mm of n3, text centered] (n4) {$u_4 : Z$}
    edge[arrin] node[above] {$1$} (n1);
  \node[thick, draw=black, ellipse, minimum height=8mm, below=12mm of n3, text centered] (n5) {$u_5 : Y$}
    edge[arrin] node[above] {$0$} (n4)
    edge[arrin] node[above] {$1$} (n2)
    edge[arrin] node[right] {$0$} (n3);
  \node[thick, draw=black, ellipse, minimum height=8mm, below left=15mm and 6mm of n5, text centered] (n6) {$u_6 : \true$}
    edge[arrin] node[below] {$1$} (n5)
    edge[arrin, bend left=4mm] node[left] {$1$} (n3)
    edge[arrin] node[left] {$0$} (n2);
  \node[thick, draw=black, ellipse, minimum height=8mm, below right=15mm and 6mm of n5, text centered] (n7) {$u_7 : \false$}
    edge[arrin] node[below] {$0$} (n5)
    edge[arrin] node[right] {$1$} (n4);
\end{tikzpicture}
\caption{A graphical representation of an \nbdd including node identifiers and their labels.}
\label{fig-nbdd-with-id}
\end{subfigure}
\hspace{5mm}
\begin{subfigure}{0.45\textwidth}
\begin{tikzpicture}[scale=0.85, transform shape]
  \node[circ, minimum size=9mm] (n1) {$X$};
  \node[circ, minimum size=9mm, below=11mm of n1] (n3) {$W$}
    edge[arrin] node[left] {$0$} (n1);
  \node[circ, minimum size=9mm, left=22mm of n3] (n2) {$Z$}
    edge[arrin] node[above] {$0$} (n1);
  \node[circ, minimum size=9mm, right=22mm of n3] (n4) {$Z$}
    edge[arrin] node[above] {$1$} (n1);
  \node[circ, minimum size=9mm, below=11mm of n3] (n5) {$Y$}
    edge[arrin] node[above] {$0$} (n4)
    edge[arrin] node[above] {$1$} (n2)
    edge[arrin] node[right] {$0$} (n3);
  \node[circ, minimum size=9mm, below left=14mm and 12mm of n5] (n6) {$\true$}
    edge[arrin] node[below] {$1$} (n5)
    edge[arrin, bend left=4mm] node[left] {$1$} (n3)
    edge[arrin] node[left] {$0$} (n2);
  \node[circ, minimum size=9mm, below right=14mm and 12mm of n5] (n7) {$\false$}
    edge[arrin] node[below] {$0$} (n5)
    edge[arrin] node[right] {$1$} (n4);
\end{tikzpicture}
\caption{The usual graphical representation of an \nbdd, where node identifiers are not included.}
\label{fig-nbdd-without-id}
\end{subfigure}
\end{center}
\caption{An \nbdd over the set of variables $\X = \{X,Y,Z,W\}$. \label{fig-nbdd}}\comm{Mikaël}{add missing edges so that all internal nodes have at least one outgoing $1$-edge and at least one outgoing $0$-edge.}
\end{figure}

\noindent
We classify nondeterministic binary decision diagrams according to two dimensions.
\begin{itemize}
\item {\bf Variable structuredness.} Let $\D$ be an \nbdd over a set of variables $\X$. Then $\D$ is \emph{free} (\nfbdd) if for every run $\pi$ of $\D$, no two distinct nodes in $\pi$ have the same label. Besides, $\D$ is {\em ordered} (\nobdd) if there exists a linear order $<$ on the set $\X$ of such that, if a node $u_1$ appears before a node $u_2$ in some run of $\D$, the label of $u_1$ is $X_1 \in \X$ and the label of $u_2$ is $X_2 \in \X$ , then $X_1 < X_2$. Notice that an $\nobdd$ is in particular an $\nfbdd$.

\item {\bf Ambiguity level.} Let $\D$ be an \nbdd over a set of variables $\X$. Then $\D$ is {\em unambiguous} (\ubdd) if for every assignment $\as \in \assign(\X)$, there exists at most one accepting run of $\D$ that is consistent with $\as$. Moreover, $\D$ is {\em deterministic}, which is referred to as \bdd \citep{L59,W04}), if for every assignment $\as \in \assign(\X)$, there exists at most one run of $\D$ that is consistent with $\as$. 
\end{itemize}
The combination of these two dimensions gives rises to 9 different classes of nondeterministic binary decision diagrams, which are shown in Figure \ref{fig-bdd}.
The most widely used models among these classes are the deterministic variants, i.e., binary decision diagrams (\bdd) \citep{L59}, free binary decision diagrams (\fbdd) \citep{FHS78,BCW80}, and ordered binary decision diagrams (\obdd) \citep{B86}. 

\comm{Marcelo}{Indicate that in the case of deterministic BDD, every internal node is assumed to have one outgoing edge with label 1 and one outgoing edge with label 0}
\comm{Antoine}{decision ratified :)}
\comm{Mikaël}{Added this in the definition of $\nbdd$, need to update the figures and the example}
\begin{figure}
\begin{center}
\begin{tabular}{l|c|c|c|}\cline{2-4}
  & Unrestricted & Free & Ordered\\\hline
 \multicolumn{1}{|l|}{Nondeterministic} & \nbdd & \nfbdd & \nobdd\\
 \multicolumn{1}{|l|}{Unambiguous} & \ubdd & \ufbdd & \uobdd\\
 \multicolumn{1}{|l|}{Deterministic} & \bdd & \fbdd & \obdd\\\hline
\end{tabular}
\end{center}
\caption{Classification of nondeterministic binary decision diagrams based on ambiguity level (nondeterministic, unambiguous, or deterministic) and variable structuredness (unrestricted, free, or ordered).\label{fig-bdd}}
\end{figure}


\begin{example}
The \nbdd in Figure \ref{fig-nbdd-without-id} is ordered because every run in it conforms to the linear order $X < W < Z < Y$. On the other hand, the following \nbdd is free but not ordered:

\end{example}

\begin{center}
\begin{tikzpicture}[scale=0.85, transform shape]
  \node[circ, minimum size=9mm] (n1) {$X$};
  \node[circ, minimum size=9mm, below=11mm of n1] (n3) {$Z$}
    edge[arrin] node[left] {$0$} (n1);
  \node[circ, minimum size=9mm, left=22mm of n3] (n2) {$Y$}
    edge[arrin] node[above] {$0$} (n1);
  \node[circ, minimum size=9mm, right=22mm of n3] (n4) {$W$}
    edge[arrin] node[above] {$1$} (n1);
  \node[circ, minimum size=9mm, below=11mm of n3] (n5) {$Y$}
    edge[arrin] node[right] {$1$} (n3);
  \node[circ, minimum size=9mm, below=11mm of n2] (n8) {$Z$}
    edge[arrin] node[left] {$1$} (n2);
  \node[circ, minimum size=9mm, below left=14mm and 12mm of n5] (n6) {$\true$}
    edge[arrin] node[below] {$1$} (n5)
    edge[arrin, bend left=4mm] node[left] {$1$} (n3)
    edge[arrin] node[left] {$0$} (n8);
  \node[circ, minimum size=9mm, below right=14mm and 12mm of n5] (n7) {$\false$}
    edge[arrin] node[below] {$0$} (n5)
    edge[arrin] node[right] {$1$} (n4);
\end{tikzpicture}
\end{center}

\paragraph*{Sharing.}
If we disallow sharing (i.e., a internal nodes having multiple paths from the root), we get decision trees.
An \nbdd $\D$ is a \emph{decision tree} (\ndt) if the underlying graph of $\D$ is a tree (ignoring the leaves).
Unlike general diagrams, we can tractably rewrite decision trees to ensure that they are free.

\paragraph*{Completing.}
We can complete OBDDs\comm{Mikaël}{even nFBDDs} to make them complete. If not complete, we can adopt the zero-suppressed semantics or not.

The definition 
TODO: cite bollig1999nondeterministic to justify the choice of notation (multiple outgoing edges chosen independently, instead of explicit OR-nodes); explain connection with the nondeterministic BDDs with or-gates, at least from amarilli2019conecting or bollig2010binary. Remark that having multiple initial nodes does not make a difference because you can just move a variable to the root.

TODO: mention NROBPs

TODO: mention smoothing, zero-suppressed semantics

and decision trees. Notice that we have not included any class of unrestricted decision trees in the figure, as the repetition of variables in a run of a decision tree can be easily eliminated, so decision trees are assumed to be free.
