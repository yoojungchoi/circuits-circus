Here, mention $k$-ambiguity, and mention revelant separation results.

Mention the Göös results and how they do not translate to diagrams


\subsection{On words}
\label{subsubsec:ufas-words}

\paragraph*{Alphabets, words, languages.}
An \emph{alphabet} is a finite set~$\Sigma$ of \emph{letters}.  A \emph{word
on~$\Sigma$} is a finite (possibly empty) sequence $\w = w_1,\ldots,w_n$ of
letters from~$\Sigma$; its \emph{length}~$|\w|$ is~$n$.  The set of all words
over~$\Sigma$ is denoted by~$\Sigma^*$.  Given a word~$\w \in \Sigma^*$ and~$i
\in \{0,\ldots,|\w|\}$, we denote by~$\w^{\leq i}$ (resp.,~$\w^{\geq i}$) the
word~$w_1,\ldots,w_i$ (resp., the word~$w_i,\ldots,w_{|\w|}$).  A \emph{
language} over~$\Sigma$ is a subset of~$\Sigma^*$. Given a language~$L$
over~$\Sigma$, its \emph{complement}, denoted~$L^\complement$, is the
language~$\Sigma^* \setminus L$.


\paragraph*{Word automata.}
A \emph{non-deterministic finite automaton} (NFA)~$\A = (Q,q_I,F,\delta)$
over~$\Sigma$ consists in a finite set~$Q$ of \emph{states}, an \emph{initial
state}~$q_I \in Q$, a set~$F \subseteq Q$ of \emph{accepting states}, and a
\emph{transition relation}~$\delta \subseteq Q \times \Sigma \times Q$.  A
\emph{run of~$\A$ on a word~$\w\in \Sigma^*$} is a sequence of states~$\rho =
q_0, q_1, \cdots, q_{|\w|}$ such that $(q_i, w_{i+1}, q_{i+1}) \in \delta$ for
every~$i \in \{0,\ldots,|\w|-1\}$. We say that~$\rho$ \emph{starts at~$q_0$}
and~\emph{ends at~$q_{|\w|}$}.  For a word~$\w$ and state~$q$, we denote
by~$\delta^*(q,\w)$ the set of states~$q'$ such that there exists a run of~$\A$
on~$\w$ starting at~$q$ and ending at~$q'$.  For a word~$\w$, we say that
\emph{$\w$ is accepted by~$\A$} if~$\delta^*(q_I,\w) \cap F \neq \emptyset$.
An \emph{accepting run} of~$\A$ on~$\w$ is a run of~$\A$ on~$\w$ that starts
at~$q_I$ and ends in a final state.
The \emph{language accepted by~$\A$}, denoted~$L(\A)$, is the set of words
over~$\Sigma$ that are accepted by~$\A$.  The automaton~$\A$ is called
\emph{unambiguous} (UFA) if for every word~$\w \in \Sigma^*$ there exist at
most one accepting run of~$\A$ on~$\w$.
%We denote by~$\ufas(\Sigma)$ (resp., $\nfas(\Sigma)$) the set of UFAs (resp.,
%NFAs) over~$\Sigma$.  
For $k \in \mathbb{N}$, we call the automaton $\A$ \emph{$k$-unambiguous} ($k$-UFA) if for every word $\w in \Sigma^*$ there exist  at most $k$ accepting runs of $\A$ on~$\w$; so $A$ is unambiguous iff it is $1$-unambiguous.

We say the automaton $\A$ is \emph{deterministic} (DFA) if, for every state $q\in Q$ and letter $a \in \Sigma$, there is at most one state $q' \in Q$ such that the transition $(q, a, q')$ is in~$\delta$. In this case, we can equivalently see $\delta$ as a partial function from $Q \times \Sigma$ to~$Q$.
Note that a deterministic automaton is necessarily unambiguous, but the converse is not true: there are some unambiguous automata which are not deterministic.

We say that a word automaton $\A$ is \emph{trimmed} if, for every state $q\in Q$, there is a word $\w$ in the language of $\A$ such that $q$ occurs in a run $\rho$ of $\A$ on~$\w$. Given a word automaton $\A$, we can clearly convert it in linear time to a trimmed automaton which is equivalent (i.e., that recognizes the same language), and this conversion preserves unambiguity, $k$-unambiguity, and determinism.

TODO define $|\A|$, for words and for trees.

\subsection{On trees}
\label{subsubsec:ufas-trees}

\paragraph*{Trees.}
Let~$\Sigma$ be an alphabet.  A \emph{$\Sigma$-tree $(T,\lambda)$} is a finite,
rooted, ordered binary tree~$T$ (all internal nodes have exactly two children,
and these are ordered) such that every node~$n$ of~$T$ is labeled by a
letter~$\lambda(n) \in \Sigma$. We say that~$T$ is the \emph{skeleton}
of~$(T,\lambda)$. We denote by~$\mathcal{T}(\Sigma)$ the set of
all~$\Sigma$-trees. We define the notions of ($\Sigma$-)tree language
over~$\Sigma$ and of complement of a tree language as expected.

\paragraph*{Tree automata.}
A \emph{non-deterministic bottom-up finite tree automaton} (NbFTA)~$\A =
(Q,F,\Delta,\iota)$ over~$\Sigma$ consists in a finite set of
\emph{states}~$Q$, a set~$F \subseteq Q$ of final states, an
\emph{initialization function} $\iota \subseteq \Sigma \times Q$, and a transition
relation $\Delta \subseteq Q \times Q \times \Sigma \times Q$.
Let~$(T,\lambda)$ be a~$\Sigma$-tree.  A \emph{run of~$\A$ on $(T,\lambda)$} is
a~$Q$-tree~$(T,\lambda')$ having same skeleton and such that:
\begin{itemize}
  \item for every leaf~$n$ of~$T$ we have that~$(\lambda(n),\lambda'(n))\in \iota$;
  \item for every internal node~$n$ of~$T$, letting~$n_1,n_2$ be the (ordered) children of~$n$,
        we have that $(\lambda'(n_1),\lambda'(n_1),\lambda(n),\lambda'(n)) \in \Delta$.
\end{itemize}
We say that run~$(T,\lambda')$ \emph{ends at state~$q$} when~$q = \lambda'(r)$
for~$r$ the root of~$T$.  The run~$(T,\lambda')$ is \emph{accepted} if,
letting~$n$ be the root of~$T$, we have~$\lambda'(n)\in F$. If there exists an
accepting run of~$\A$ on~$(T,\lambda)$ then we say that~$(T,\lambda)$ is
accepted by~$\A$. We define~$L(\A)$ as usual. Automaton~$\A$ is \emph{unambiguous} (UbFTA) if
for any~$\Sigma$-tree~$(T,\lambda)$, there exists at most one accepting run
of~$\A$ on~$(T,\lambda)$.
For $k \in \mathbb{N}$, we say that $\A$ is \emph{$k$-unambiguous} ($k$-UbFTA) if the same holds but there are at most $k$ accepting runs; so $1$-unambiguity is equivalent to unambiguity.

We say that $\A$ is a \emph{deterministic bottom-up finite tree automaton}
(DbFTA) if, for each pair of states $(q_1, q_2) \in Q$ and letter $a \in
\Sigma$, there is at most one state $q$ such that $(q_1, q_2, a, q) \in \delta$.
In this case, we can equivalently see $\delta$ as a partial function from $Q
\times Q \times \Sigma$ to $Q$.

We say that $\A$ is \emph{trimmed} if, for every state $q \in Q$, there is a
$\Sigma$-tree $(T,\lambda)$ in the language accepted by $\A$ and a run $\rho$ of
$\A$ on $(T, \lambda)$ in which $q$ appears. Given a tree automaton, we can
convert it in linear time to an automaton which is equivalent (recognizes the
same language) and is trimmed, and this conversion preserves unambiguity,
$k$-unambiguity, and determinism.

Note that a trimmed UbFTA $\A$ over~$\Sigma$ satisfies the following stronger
property (*):
%is \emph{strongly unambiguous} if 
for every
$\Sigma$-tree~$(T,\lambda)$ and state~$q \in Q$ there is at most one run
of~$\A$ on~$(T,\lambda)$ that ends at~$q$. This property may not be satisfied if
$\A$ is not trimmed if $q$ is not final and informally cannot lead to a finite
state.

\subsection{Computing circuits from automata}

We now explain how to compute a circuit from an automaton. We first give the
construction for the most general formalism possible (tree automata without any
restrictions), before specializing it and noticing the resulting properties on
the circuit. All the circuits that we construct will be decomposable and in
negation normal form (DNNF), and they were all be structured (for circuits) or
ordered (for decision diagrams).

We assume that the alphabet used by automata is $\Sigma = \{0, 1\}$; we will
later explain why other alphabets can also be handled.

\begin{definition}
  Let $\A$ be a FWA over alphabet $\Sigma = \{0, 1\}$, and let $n \in \NN$. The
  \emph{provenance} of~$\A$ on~$n$ is a Boolean function $\phi_{\A, n}$
  over variables
  $\X = \{x_1, \ldots, x_n\}$ such that, for any assignment $\ba$
  of~$\X$, letting $w_\ba$ be the corresponding word of length $n$ over
  $\Sigma$, we have that $\ba$ satisfies $\phi_{\A, n}$ iff $\A$ accepts $w_\ba$.
  A \emph{provenance circuit} of~$\A$ on~$n$ is a circuit representing
  $\phi_{\A, n}$.

  Likewise, if $\A$ is a bFTA over alphabet $\Sigma = \{0, 1\}$ and $T$ is an
  unlabeled full binary tree, the \emph{provenance} of~$\A$ on~$T$ is a Boolean
  function $\phi_{\A, T}$ whose variables $\X$ are the nodes of~$T$ such that,
  for any assignment $\ba$ of~$\X$, letting $T_\ba$ be the $\Sigma$-tree
  obtained from~$T$ by labeling the nodes according to~$\ba$, we have that $\ba$
  satisfies $\phi{\A, T}$ iff $\A$ accepts $T_\ba$. We define a \emph{provenance
  circuit} in the expected way.
\end{definition}

We show in this section the following two results on word automata and tree
automata:

\begin{proposition}
  \label{prp:wordprov}
  Let $\A$ be a FWA over alphabet $\Sigma = \{0, 1\}$ and let $n \in \NN$. We
  can build in time $O(|\A| \times n)$ a provenance circuit $C$ of~$\A$ on~$n$
  which is an nOBDD with variable order $x_1, \ldots, x_n$.
  Further, if $\A$ is unambiguous then $C$ is an uOBDD, and if $\A$ is
  deterministic then $C$ is an OBDD.
\end{proposition}

\begin{proposition}
  \label{prp:treeprov}
  Let $\A$ be a bFTA over alphabet $\Sigma = \{0, 1\}$ and let $T$ be an
  unlabeled full binary tree. We
  can build in time $O(|\A| \times |T|)$ a provenance circuit $C$ of~$\A$ on~$n$
  which is a structured DNNF (SDNNF), together with its vtree.
  Further, if $\A$ is unambiguous then $C$ is a d-SDNNF.
\end{proposition}

TODO: no correspondence when $\A$ is deterministic, except the weird notion of
``upwards-determinism'' from \cite{amarilli2017circuit}...

We briefly comment on the relationship of these results
to~\cite{amarilli2015provenance}. In the latter work, the provenance for
automata is defined on an alphabet consisting of a fixed part together with a
Boolean annotation. For instance, for word automata, the alphabet is $\Sigma
\times \{0, 1\}$, and the provenance is defined on an input word of~$\Sigma^*$,
to describe which of the Boolean annotations of the word are accepted. The
results above, with alphabet $\{0, 1\}$, allow us to recapture this setting.
Indeed, we can modify automata working on a larger alphabet to first read a
binary representation of their letter followed by the Boolean annotation.
Applying the results above, and fixing the inputs corresponding to letters to
the intended values, gives us the provenance circuit in the sense
of~\cite{amarilli2015provenance}. This uses the tractability of setting a
variable to a value, aka conditioning (see Section~\ref{sec:operations})

We first prove Proposition~\ref{prp:wordprov} as a warmup:

\begin{proof}[Proof of Proposition~\ref{prp:wordprov}]
  We assume that the input automaton $\A$ is \emph{complete} in the sense that,
  for every state $q$ and every letter $b \in \Sigma$, there is at least one
  transition for letter $b$ on state~$q$. We can clearly make $\A$ complete in
  linear time up to adding a sink state.

  We build a diagram with decision nodes $n_{i,q}$ for each $1 \leq i \leq n$ and
  for each state~$q$, each being a decision on variable $x_i$.
  The initial node will be $g_{1, q_I}$ for $q_I$ the
  initial state of the automaton. We also have sinks $n_{n+1,q}$ for each
  state~$q$: they
  are 0-sinks for non-final $q$ and 1-sinks for final~$q$.

  Now, for every $1 \leq i \leq n$, we add the following edges: for each
  transition from a state $q$ to state $q'$ when reading symbol $b \in \Sigma$,
  we add an edge to $n_{i,q}$ labeled~$b$ to $n_{i+1,q'}$.

  The construction satisfies the time bound. We modify the result in linear time
  to trim it, removing all nodes that do not have a path from the starting node.
  The result is an nOBDD, because
  each decision node has at least one outgoing 0-edge and 1-edge, and the
  variables are ordered in the right way. Note that it is also complete in the
  sense that it tests all of its variables (TODO define if not defined).

  We show correctness by finite induction: for every $0 \leq i \leq n$, given
  a word $w$ of length $i$, the reachable nodes $n_{i,q}$ by the partial
  valuation corresponding to~$w$ are those for the states $q$ that we can reach
  when reading~$w$. The base case is immediate: when reading the empty word, we
  can get precisely to the initial state. Now, for the induction step, if the
  states that we can reach when reading a word of length $i$ are correct, then
  when reading an extra letter $b \in \Sigma$ we can go precisely to the states
  having a transition labeled $b$ from the previously reachable states, so the
  result is correct. The finite induction applied to $i=n$ confirms that we can
  reach a final state (i.e., the word is accepted) iff we can reach a 1-sink.

  Now, observe that if the automaton is unambiguous, then for any valuation
  there is at most one way to reach the 1-sink, because there is at most one
  accepting run of the automaton on the corresponding word. Further, if the
  automaton is deterministic, then for any valuation there is at most one
  consistent path, because each decision node has at most one outgoing 0-edge
  and 1-edge.
\end{proof}

We next prove Proposition~\ref{prp:treeprov}. To do this, we first need to do a
small adjustment: v-trees are defined so that the variables are at the leaves,
but tree automata read labels on all tree nodes, including internal nodes.

\begin{definition}
  Let $T$ be a finite rooted ordered binary tree with nodes $N$. Its
  \emph{leaf-push}
  is the finite rooted ordered binary tree $T'$ obtained as follows: replace
  each internal node $n$ with children $n_1$ and $n_2$ by an internal node $n'$
  having as children one leaf $n$ and an internal node $n''$ having as children
  $n_1$ and $n_2$.

  Note that the leaves of $T'$ are precisely in bijection with the nodes of~$T$.
\end{definition}

\begin{proof}[Proof of Proposition~\ref{prp:treeprov}]
  We do not assume this time that the input tree automaton $\A$ is complete, but
  must instead assume that it is \emph{trimmed}, so as to satisfy property (*)
  above \comm{antoine}{note: this difference with the case of word automata is a bit
  inelegant}

  We build a circuit with $\lor$-gates $g_{n,q}$ for each tree node~$n$ of the
  input tree~$T$.
  The output gate will be a $\lor$-gate $g$ doing the disjunction of all the
  $g_{r,q}$ for $q$ final and for $r$ the root of~$T$.

  Now, for every leaf $n$ of the input tree~$T$, we add
  a variable gate for~$n$ as an input of the gate $g_{n,q}$ for each $q \in \iota(1)$, and
  we add a gate for the negation of~$n$ as an input of the gate $g_{n,q}$ for
  each $q \in \iota(0)$.

  For every internal node $n$ of the input tree $T$ with children $n_1$ and
  $n_2$, for every pair of states $q_1$ and $q_2$, for every $b \in \{0, 1\}$,
  for every state $q$ having a transition from $q_1, q_2, b$, we add as input to
  $g_{n,q}$ a $\land$-gate having as first input either $n$ or $\neg n$
  depending on whether $b=1$ or $b=0$, and as second input a $\land$-gate having
  as first input $g_{n_1,q_1}$ and as second input $g_{n_2,q_2}$.

  The construction satisfies the time bound. TODO trim if necessary. Note that
  the circuit is clearly in NNF. We claim that it is decomposable, and that it
  is structured by a v-tre which is the leaf-push of the tree~$T$. Indeed, we
  can easily show by bottom-up induction that the domain of each gate $g_{n,q}$
  is a subset of the nodes of the subtree of~$T$ rooted at~$n$ (including~$n$),
  so that the $\land$-gates are indeed structured according to the leaf-push
  (TODO details if necessary).
  Note that the circuit is smooth (TODO define if not defined) if we remove the
  unsatisfiable gates.

  We show correctness again by finite induction: for every tree node~$n$ of~$T$, given
  a labeling of the subtree~$T_n$ of~$T$ rooted at~$n$,
  the gates $g_{n,q}$ satisfied by the corresponding partial valuation are
  precisely those for the states $q$ that we can reach
  when reading the labeling of~$T_n$ in~$\A$.
  The base case corresponds to leaves, where indeed the gates $g_{n,q}$ that are
  satisfied follow the initial function $\iota$ (TODO details).

  Now, for the induction step, considering an internal node~$n$ of~$T$ with
  children $n_1$ and $n_2$, if we assume that the
  states that we can reach when reading a labeling $\lambda_1$ of the tree
  $T_{n_1}$ and when reading a labeling $\lambda_2$ of the tree $T_{n_2}$ are
  correct, let us show the same when reading a labeling $\lambda$ of~$T_n$
  defined as $\lambda_1$ on~$T_{n_1}$ and $\lambda_2$ on $T_{n_2}$ and $b \in
  \Sigma$ on~$n$. The gates $g_{n,q}$ that are satisfied are precisely those for
  which there is a transition from $q_1$ and $q_2$ and $b$ to~$q$ and
  $g_{n_1,q_1}$ and $g_{n_2,q_2}$ are satisfied, allowing us to conclude from
  the induction hypothesis (TODO details). Hence, using the induction result on
  the root $r$ of~$T$ (where $T_r = T$), we conclude that the output gate of the
  circuit is true on a valuation iff there is a final state of~$\A$ that we can
  reach on the labeling of~$T$ according to this valuation.

  Now, observe that if the automaton is unambiguous, then each $\lor$-gate is
  deterministic. Indeed, if the output gate had two inputs that are mutually
  satisfiable, then it witnesses the existence of a labeling of the tree~$T$
  on which $\A$ reaches two different final states, contradicting the
  unambiguity of~$\A$. Likewise, if a gate $g_{n,q}$ has two inputs that are
  mutually satisfiable, as they are satisfied by the same valuation the literal
  on~$n$ must be the same so it must be the case that we can simultaneously
  satisfy $g_{n_1, q_1}$ and $g_{n_2, q_2}$, and $g_{n_1,q_1'}$ and
  $g_{n_2,q_2'}$, for two 2-tuples $(q_1,q_2) \neq (q_1',q_2')$. This witnesses
  that, on this labeling of~$T_n$, the automaton has two distinct runs leading
  to state~$q$ at the root. This is a contradiction of property (*).
\end{proof}

\subsection{From DNFs to automata}

We last mention a result of a different kind about the translation from Boolean
formalisms to automata formalisms: given a DNF, we can easily translate them to
a (non-deterministic) automaton checking that they are satisfied. (TODO) If the
DNF is unambiguous (TODO define), then the automaton also is.

\subsection{k-unambiguity}

Motivated by the natural definition of $k$-unambiguity for automata on words on
trees, we can define a similar notion for diagrams and circuits.

\begin{definition}
  An nOBDD (TODO: without order?) is \emph{$k$-unambiguous} for $k\in\NN$,
  called an \emph{$k$-uOBDD}, if for every valuation there is at most $k$
  distinct paths going from the initial node to a 1-sink. In particular, a
  $1$-uOBDD is exactly an uOBDD.
\end{definition}

For circuits, this is a bit less pleasant, and we must define a \emph{trace} of
a circuit:

\begin{definition}
  A \emph{trace} of a DNNF $C$ is a DAG defined in the following inductive way:
  \begin{itemize}
    \item The output gate of~$C$ is the root of the trace
    \item For every $\land$-gate $g$ with 2 inputs which is in the trace, both
      of the inputs are in the trace.
    \item For every $\lor$-gate $g$, exactly one of its inputs is in the trace.
      (In particular, a $\lor$-gate with no inputs cannot occur in the trace.)
  \end{itemize}
  The trace is \emph{consistent} with a valuation if the literals that it
  contains as leaves are consistent with the valuation (TODO tidy up).
\end{definition}

Note that, for every valuation, if the circuit has a trace for that valuation
then it is a satisfying valuation: the trace witnesses that the circuit
evaluates to true. Conversely, given a satisfying valuation, we can
build a trace by selecting gates that evaluate to true under the
valuation.

Further, let us now assume that the circuit is \emph{trimmed} in the following
sense \comm{antoine}{Push up}:
\begin{itemize}
  \item Every gate has a directed path to the output
  \item There are no unsatisfiable gates
\end{itemize}
(Note that this imposes that the circuit is satisfiable)
Then, given a valuation and a gate $g$ of the circuit, we
can build a partial trace for that valuation rooted at~$g$ iff $g$ evaluates to
true under the valuation.
We claim that a DNNF is deterministic iff, for every valuation, it has at most
one trace. Indeed, if it is deterministic, then for every $\lor$-gate in the
trace we have only one true input to select. Conversely, if there is a valuation
and gate $g$ witnessing a violation of determinism, then we have two distinct
partial traces rooted at~$g$, and we can conclude that there are two distinct
traces of the circuit using the directed path from~$g$ to the output: note that
when we traverse a $\land$-gate then its other input is satisfiable because the
circuit is trimmed, and thanks to decomposability we can freely choose the
valuation of the variables that will make this other input satisfied.

\begin{definition}
  A trimmed DNNF is \emph{$k$-deterministic} (or \emph{$k$-d-DNNF})
  for $k \in \NN$ if, for every assignment, it has at most $k$ traces.
\end{definition}

With these notions, the statements of Proposition~\ref{prp:wordprov} and
Proposition~\ref{prp:treeprov} can be extended to say that a $k$-unambiguous
automata gets converted to a $k$-uOBDD (for words) or $k$-d-DNNF (for trees).
(TODO)

\subsection{Normal form and width}

TODO: Mention the definition of width for diagrams and for circuits, and the
normal form where ANDs and ORs alternate.


\antoine{We have to understand what SDDs mean in terms of tree automata.
Definition (in the context where automata only read labels on leaves not on
internal nodes), a tree automaton is strongly deterministic if it has only one
final state and if for any state $q$, for any distinct pairs $(q_1, q_2)$ and $(q_1',
q_2')$ leading to~$q$, then $q_1$ and $q_1'$ are not compatible, where two
states are compatible if there is a tree on which the automaton has a run
labeling the root with $q_1$ and a run labeling the root with $q_1'$ (in
particular it is the case if $q_1 = q_1'$). Bottom-up deterministic automata are
not necessarily strongly deterministic, but deterministic top-down automata are
strongly deterministic (but it's not longer clear you can do the standard finite
argument showing that top-down deterministic is weaker). Big question: can you
convert any tree automata to a strongly deterministic tree automaton?}

