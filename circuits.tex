A \emph{circuit} $C$ consists of a rooted directed acyclic graph $(G, W)$ whose vertices $G$ are called \emph{gates}, whose edges $W \subseteq G \times G$ are called \emph{wires}, and whose root node is called the \emph{output gate}, denoted $g_0$. 
An \emph{input} of a gate $g \in G$ is a gate $g' \in G$ that has a wire to~$g$, i.e., we have $(g', g) \in W$.
Fixing a set $\X$ of variables, the gates of~$C$ can be of several kinds:
\begin{itemize}
\item A \emph{variable}, annotated with a variable of~$\X$, which must then have no input;
\item A \emph{negation gate}, which must have exactly one input;
\item An \emph{$\land$-gate} or \emph{$\lor$-gate}, which can have an arbitrary number (including zero) of inputs.
\end{itemize}
The circuit $C$ over variables $\X$ represents a Boolean function over $\X$ in the following way.
Given a valuation $\ba\colon \X \to \{0, 1\}$, we extend it to give a Boolean value to all gates of the circuit by bottom-up induction:
\begin{itemize}
    \item The value of a variable gate $g$ annotated with variable $X$ is $\ba(g) \colonequals \ba(X)$;
    \item The value of a negation gate $g$ is $\ba(g) \colonequals 1 - \ba(g')$, where $g'$ is the input of~$g$;
    \item The value of an $\land$-gate $g$ (resp., $\lor$-gate $g$) with input gates $g_1, \ldots, g_n$ is $\ba(g) \colonequals \bigwedge_i \ba(g_i)$ (resp., $\ba(g) \colonequals \bigvee_i \ba(g_i)$).
\end{itemize}
In particular, note that the evaluation of $\land$-gates with no inputs and $\lor$-gates with no inputs is always $1$ and $0$ respectively (i.e., the neutral element of the corresponding operation), so these can be used as constant gates. The Boolean function defined by $C$ is then the one that maps valuations $\ba$ to the value $\ba(g_0)$ of the output gate of the circuit.

The size~$|C|$ of a circuit is its number
of wires.  For a gate~$g$ of~$C$, we denote by~$\vars(g)$ the set of variables
that have a directed path to~$g$ in~$C$, and we denote by~$C_g$ the Boolean
circuit over~$\X$ whose output gate is~$g$. 

In the rest of this document, we will always consider circuits in \emph{negation normal form} (NNF), where negation gates are always applied to variables; formally, for any negation gate $g$, then its one input must be a variable gate. We can equivalently see NNF circuits as positive circuits (circuits without negation) defined directly on the literals (i.e., the variables and their negations).

We classify circuits according to three
dimensions.

\begin{itemize}
    \item \textbf{Variable structuredness.} A circuit is called \emph{decomposable} if, intuitively, $\land$-gates partition the variables into disjoint sets.
Formally, an $\land$-gate~$g$ of~$C$ is
\emph{decomposable} if it has exactly zero (i.e., it is a constant $1$-gate) or two input gates and, in case it has two input gates $g_1\neq g_2$ we
have~$\vars(g_1) \cap \vars(g_2) = \emptyset$. A circuit $C$ is \emph{decomposable} if every $\land$-gate of~$C$ is.
    Note that this is a syntactic condition that can easily be checked in time $O(|C||\X|)$. A decomposable NNF circuit is called a \emph{DNNF}. We point out that DNNFs are sometimes defined without the restriction that $\land$-nodes always have zero or two inputs; we impose this for convenience, and this is without much loss of generality as this can be enforced in linear time.
    
    Further, a DNNF is \emph{structured} if the partitions defined by the $\land$-gates are compatible, in the following sense.
A \emph{v-tree} over the set of variables $\X$ is a rooted full binary tree $T$ whose leaves are in bijection with $\X$.  We
always identify each leaf with the associated element of $\X$. For a node~$n\in T$, we abuse notation and denote by~$\vars(n)$ the set of variables in the subtree rooted at~$n$. A DNNF $D$ is \emph{structured by the v-tree $T$} if there exists a mapping $\rho$ labeling each $\land$-gate of $g$ with a node of $T$ that satisfies the following: for
every $\land$-gate $g$ of $D$ with two inputs $g_1, g_2$, the node $\rho(g)$ \emph{structures} $g$, i.e.,
there exist two distinct children $n_1$, $n_2$ of $\rho(g)$ such that $\vars(g_i) \subseteq \vars(T_{n_i} )$ for $i=1,2$. A DNNF is \emph{structured}, written \emph{SDNNF}, if there exists a v-tree that structures it. 


\item \textbf{Ambiguity level.} A circuit is \emph{unambiguous}, which is (unfortunately) called for historical reasons \emph{deterministic}, if, intuitively, the inputs to $\lor$-gates are mutually exclusive. Formally,
an~$\lor$-gate~$g$ of~$C$ is \emph{deterministic} if
for every pair~$g_1\neq g_2$ of input gates of~$g$, the Boolean functions
over~$\X$ captured by~$C_{g_1}$ and~$C_{g_2}$ are disjoint; that is, we
have~$\sat(C_{g_1}) \cap \sat(C_{g_2}) = \emptyset$.  We call~$C$
\emph{deterministic} if each~$\lor$-gate is.
    
A stronger notion of determinism is that of determinism in the sense of BDDs, which we call \emph{decision} to distinguish it from determinism. An $\lor$-gate is \emph{decision} if it has no inputs or has exactly two inputs $g_0$ and $g_1$ and there is a variable $X$ such that $g_1$ is an $\land$-gate with $X$ as input, and $g_0$ is an $\land$-gate with a negation gate of $X$ as input. The circuit $C$ is \emph{decision} if all $\lor$-gates are decision. Note that being decision is a syntactic condition, and a decision circuit is always deterministic.
\end{itemize}

When considering combinations of the two, people typically assume decomposability. 
A circuit which is both decomposable and deterministic is called a \emph{d-DNNF}, and if it is in addition structured then we have a d-SDNNF. 
 A circuit which is both decomposable and decision is called a \emph{dec-DNNF}, and \emph{dec-SDNNF} if it is structured.


\begin{example} \label{ex:class}
The following example is taken from~\cite{arenas2023complexity}.
   We want to classify papers submitted to a conference as rejected
(Boolean value $0$) or accepted (Boolean value $1$). Papers are
described by Boolean features \feat{fg}, \feat{dtr}, \feat{nf} and \feat{na},
which stand for ``follows guidelines", ``deep theoretical result", ``new
framework" and ``nice applications", respectively.  The Boolean classifier
for the papers is given by the Boolean circuit in
Figure~\ref{fig:ddbc-exa}. The input of this circuit are the
features \feat{fg}, \feat{dtr}, \feat{nf} and \feat{na}, each of which
can take value either $0$ or $1$, depending on whether the feature is
present~($1$) or absent~($0$). The nodes with labels~$\neg$, $\lor$ or~$\land$ are logic gates, and the associated Boolean value of
each one of them depends on the logical connective represented by its
label and the Boolean values of its inputs. The output
value
of the circuit is given by
the top node in the figure.

The Boolean circuit in
Figure~\ref{fig:ddbc-exa} is decomposable, because each
$\land$-gate has two inputs, and the sets of features of its inputs are pairwise
disjoint. For instance, in the case of the top node in
Figure~\ref{fig:ddbc-exa}, the left-hand side input has
$\{\feat{fg}\}$ as its set of features, while its right-hand side
input has $\{\feat{dtr}, \feat{nf}, \feat{na}\}$ as its set of
features, which are disjoint. Also, this circuit is deterministic as 
for every $\lor$-gate two (or more) of its inputs cannot be given
value 1 by the same Boolean assignment for the features. For instance,
in the case of the only $\lor$-gate in Figure~\ref{fig:ddbc-exa}, if a
Boolean assignment for the features gives value 1 to its left-hand side
input, then feature \feat{dtr} has to be given value 1 and, thus, such
an assignment gives value $0$ to the right-hand side input of the~$\lor$-gate.
In the same way, it can be seen that if a Boolean assignment for the features
gives value 1 to the right-hand side input of this $\lor$-gate, then it gives
value $0$ to its left-hand side input.
\qed
\end{example}

\begin{figure}
\begin{center}
\begin{center}
\begin{tikzpicture}
  \node[circ, minimum size=7mm, inner  sep=-2] (n1) {\feat{dtr}};
  \node[circ, right=12mm of n1, minimum size=7mm] (n2) {\feat{nf}};
  \node[circ, above=1.3mm of n2, minimum size=7mm] (nneg) {$\neg$}
    edge[arrin] (n1);
  \node[circ, right=12mm of n2, minimum size=7mm] (n3) {\feat{na}};
  \node[circ, right=6mm of nneg, minimum size=7mm] (nbla) {$\land$}
  edge[arrin] (n2)
  edge[arrin] (n3);
  \node[circ, above=10mm of n3, minimum size=7mm] (n4) {$\land$}
  edge[arrin] (nneg)
  edge[arrin] (nbla);
  \node[circ, above=18mm of n2, minimum size=7mm] (n5) {$\lor$}
  edge[arrin] (n4)
  edge[arrin] (n1);
  \node[circ, above=27mm of n1, minimum size=7mm] (n6) {$\land$}
  edge[arrin] (n5);
  \node[circw, left=12mm of n5, minimum size=7mm] (n5a) {};
  \node[circ, left=12mm of n5a, minimum size=7mm, inner sep=-2] (n0) {\feat{fg}}
  edge[arrout] (n6);
\end{tikzpicture}
\end{center}
\caption{A deterministic and decomposable Boolean Circuit as a classifier. \label{fig:ddbc-exa}}
\end{center}
\end{figure}

\subparagraph*{SDD}
\todo[inline]{TODO: YooJung}
% \todo[inline]{Antoine: move this to the circuit section, keep this definition}
There are intermediate classes between determinism and decomposability. First,
we impose structuredness. Then we impose:
\begin{itemize}
    \item Strong determinism: all or-gates have as inputs some ANDs, and the LHS of the ANDs are mutually exclusive
    \item Exhaustiveness: Further, the disjunction of the LHSs is true (= the LHS partition the space)
\end{itemize}
These classes typically only make sense for structured classes (indeed otherwise
the first one is trivial). So an SDD is a structured circuit which is
strongly deterministic and exhaustive.
\todo[inline]{Antoine: not for here but to keep in mind: is this related to classes of tree languages? or to classes of CFGs?}

As for decision diagrams, one can also be interested in whether the circuits
allow sharing, or if they check all the variables:

\paragraph*{Sharing.}
When the underlying graph of a circuit is a tree (noting that each variable
can occur multiple times), then we call the result a \emph{Boolean formula}
(not to be confused with a Boolean function). If the circuit is in NNF,
then the Boolean formula is in NNF. Examples of Boolean formulas in NNF
include Boolean formulas in \emph{conjunctive normal form} (CNF), a
conjunction of disjunction of literals; and Boolean formulas in
\emph{disjunctive normal form} (DNF), a disjunction of conjunctions of
literals. Note that Boolean formulas may still contain several occurrences
of the same variable or literal; when we further impose that the circuit
has a tree structure and that there is only one gate for each variable, we
obtain what is called a \emph{read-once formula}.

\paragraph*{Smoothness and Zero-suppressed semantics.}
An $\lor$-gate $g$ of a circuit $C$ is \emph{smooth} if for every input $g'$ of
$g$ we have $\vars(g)=\vars(g')$, and $C$ is called \emph{smooth} if all its
$\lor$-gates are smooth. Smoothness is the circuit analogue of
\emph{completeness} for decision diagrams. \emph{Smoothing} is the process of
taking a circuit~$C$ as input, and constructing a circuit~$C'$ that is
equivalent to $C$ and that is smooth. This can be done naïvely in quadratic
time as follows. Letting $\X$ be the variables, compute in time $O(|C||\X|)$
the set $\vars(g)$ for every $g\in \X$. Then, for every $\lor$-gate $g'$ and
input gate $g'$ such that $\vars(g') \subsetneq \vars(g')$, replace $g'$ by an
$\land$-gate of $g'$ and of $\bigwedge_{X \in \vars(g)\setminus \vars(g')} (X
\lor \lnot X)$. It is clear that the resulting circuit is equivalent and
smooth, and that this can be done in $O(|C||\X|)$. Furtheremore, if $C$ was
decomposable, then $C'$ also is, and the same is true for being in NNF. We note
that smoothing can be done more efficiently for structured circuit
classes~\citep{DBLP:conf/nips/ShihBBA19}. As for decision diagrams, one can
also define a \emph{zero-suppressed semantics} for circuits: the idea is that a
variable that is not mentioned has to be assigned the value $0$ (as opposed to
being allowed to be $0$ or $1$). We refer to~\cite{amarilli2017circuit} for the
formal definitions.

\todo[inline]{YooJung: connections to probabilistic circuits? could be moved elsewhere too}
\todo[inline]{Antoine: I agree we can try putting it here}

