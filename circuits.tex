A \emph{circuit} $C$ consists of a rooted directed acyclic graph $(G, W)$ whose vertices $G$ are called \emph{gates}, whose edges $W \subseteq G \times G$ are called \emph{wires}, and whose root node is called the \emph{output gate}, denoted $g_0$. 
An \emph{input} of a gate $g \in G$ is a gate $g' \in G$ that has a wire to~$g$, i.e., we have $(g', g) \in W$.
Fixing a set $\X$ of variables, the gates of~$C$ can be of several kinds:
\begin{itemize}
\item A \emph{variable}, annotated with a variable of~$\X$, which must then have no input;
\item A \emph{negation gate}, which must have exactly one input;
\item An \emph{$\land$-gate} or \emph{$\lor$-gate}.
\end{itemize}
The circuit $C$ over variables $\X$ represents a Boolean function over $\X$ in the following way.
Given a valuation $\ba\colon \X \to \{0, 1\}$, we extend it to give a Boolean value to all gates of the circuit by bottom-up induction:
\begin{itemize}
    \item The value of a variable gate $g$ annotated with variable $X$ is $\ba(g) \colonequals \ba(X)$;
    \item The value of a negation gate $g$ is $\ba(g) \colonequals 1 - \ba(g')$, where $g'$ is the input of~$g$;
    \item The value of an $\land$-gate $g$ (resp., $\lor$-gate $g$) with input gates $g_1, \ldots, g_n$ is $\ba(g) \colonequals \bigwedge_i \ba(g_i)$ (resp., $\ba(g) \colonequals \bigvee_i \ba(g_i)$).
\end{itemize}
In particular, note that the evaluation of $\land$-gates with no inputs and $\lor$-gates with no inputs is always $1$ and $0$ respectively (i.e., the neutral element of the corresponding operation), so these can be used as constant gates. The Boolean function defined by $C$ is then the one that maps valuations $\ba$ to the value $\ba(g_0)$ of the output gate of the circuit.

The size~$|C|$ of a circuit is its number
of wires.  For a gate~$g$ of~$C$, we denote by~$\vars(g)$ the set of variables
that have a directed path to~$g$ in~$C$, and we denote by~$C_g$ the Boolean
circuit over~$\X$ whose output gate is~$g$. 

In the rest of this document, we will always consider circuits in \emph{negation normal form} (NNF), where negation gates are always applied to variables; formally, for any negation gate $g$, then its one input must be a variable gate. We can equivalently see NNF circuits as positive circuits (circuits without negation) defined directly on the literals (i.e., the variables and their negations).

We classify circuits according to three
dimensions.

\begin{itemize}
    \item \textbf{Variable structuredness.} A circuit is called \emph{decomposable} if, intuitively, $\land$-gates partition the variables into disjoint sets.
Formally, an $\land$-gate~$g$ of~$C$ is
\emph{decomposable} if it has exactly zero (i.e., it is a constant $1$-gate) or two input gates and, in case it has two input gates $g_1\neq g_2$ we
have~$\vars(g_1) \cap \vars(g_2) = \emptyset$. A circuit $C$ is \emph{decomposable} if every $\land$-gate of~$C$ is.
    Note that this is a syntactic condition that can easily be checked in time $O(|C||\X|)$. A decomposable NNF circuit is called a \emph{DNNF}. We point out that DNNFs are sometimes defined without the restriction that $\land$-nodes always have zero or two inputs; we impose this for convenience, and this if without much loss of generality as this can be enforced in linear time.
    
    Further, a DNNF is \emph{structured} if the partition defined by the $\land$-gates is always the same, in the following sense.
A \emph{v-tree} over the set of variables $\X$ is a rooted full binary tree $T$ whose leaves are in bijection with $\X$.  We
always identify each leaf with the associated element of $\X$. For a node~$n\in T$, we abuse notation and denote by~$\vars(n)$ the set of variables in the subtree rooted at~$n$. A DNNF $D$ is \emph{structured by the v-tree $T$} if there exists a mapping $\rho$ labeling each $\land$-gate of $g$ with a node of $T$ that satisfies the following: for
every $\land$-gate $g$ of $D$ with two inputs $g_1, g_2$, the node $\rho(g)$ \emph{structures} $g$, i.e.,
there exist two distinct children $n_1$, $n_2$ of $\rho(g)$ such that $\vars(g_i) \subseteq \vars(T_{n_i} )$ for $i=1,2$. A DNNF is \emph{structured}, written \emph{SDNNF}, if there exists a v-tree that structures it. Note that, given as input a DNNF, it is possible to determine in \todo[inline]{$O(TODO)$} if it is structured, and if so compute an associated v-tree and mapping $\rho$.
% \todo[inline]{Say something to justify the arity-two restriction:  is this only for convenience to define v-trees?}
%M: added comment where we define decomposability


\item \textbf{Ambiguity level.} A circuit is \emph{unambiguous}, which is (unfortunately) called for historical reasons \emph{deterministic}, if, intuitively, the inputs to $\lor$-gates are mutually exclusive. Formally,
an~$\lor$-gate~$g$ of~$C$ is \emph{deterministic} if
for every pair~$g_1\neq g_2$ of input gates of~$g$, the Boolean functions
over~$\X$ captured by~$C_{g_1}$ and~$C_{g_2}$ are disjoint; that is, we
have~$\sat(C_{g_1}) \cap \sat(C_{g_2}) = \emptyset$.  We call~$C$
\emph{deterministic} if each~$\lor$-gate is. A circuit which is both decomposable and deterministic is called \emph{d-DNNF}, and if it is in addition structured then we have a d-SDNNF. 
A stronger notion of determinism is that of determinism in the sense of BDDs, which we call \emph{decision} to distinguish it from determinism. An $\lor$-gate is \emph{decision} if it has no inputs or has exactly two inputs $g_0$ and $g_1$ and there is a variable $X$ such that $g_1$ is an $\land$-gate with $X$ as input, and $g_0$ is an $\land$-gate with a negation gate of $X$ as input. The circuit $C$ is \emph{decision} if all $\lor$-gates are decision. Note that being decision is a syntactic condition, and a decision circuit is always deterministic. A circuit which is both decomposable and decision is called \emph{dec-DNNF}, and \emph{dec-SDNNF} if it is structured.
\end{itemize}

We point out that determinism is a semantic
condition rather than a syntactic one. In
fact, we can show that determining whether an
input DNNF circuit is deterministic is an
coNP-complete problem. To the best of our knowledge, we are the first to formally prove this:

\begin{proposition}
The following decision problem is
coNP-complete: given as input a DNNF $C$,
decide whether it is deterministic.
\end{proposition}
\begin{proof}
  \mikael{todo quickly explain the proof of TODO that
  determining if a given $\lor$-gate of a
  circuit is deterministic is coNP-complete,
  explain that this does not imply the claim
  of this proposition, then show a proper
proof.}
\end{proof}


\paragraph*{Connections between diagrams and
circuits.} 
Circuits generalize diagrams, with
the following general linear-time translation
from diagrams to circuits. We replace true
leaves and false leaves respectively by
constant true and false gates, i.e.,
$\lor$-gates with no inputs and $\land$-gates
with no inputs respectively. We replace
internal nodes $n$ on a variable $X$ with a
gate defined like in the decision of decision
gates above; formally, let $g^1_0,\ldots, g^{k_0}_0$ (resp., $g^1_1,\ldots, g^{k_1}_1$)
be the translations of the nodes and
to which $n$ had edges labeled
with $0$ (resp., with $1$).
Construct a gate $g_0$ (resp., $g_1$) to be an $\lor$-gate of the gates $g^1_0,\ldots, g^{k_0}_0$ (resp., of $g^1_1,\ldots, g^{k_1}_1$).
We then translate
$n$ to a $\lor$-gate whose inputs are an
$\land$-gate conjoining $g_1$ and $X$,
and an $\land$-gate conjoining $g_0$
and a negation gate of~$X$. Last, the output (root) gate of the
circuit is the translation of the root of the
BDD.

It is clear that this translation runs in
linear time and produces a circuit with the
same semantics, i.e., that represents the same
Boolean function, as the BDD. What is more:
\begin{itemize}
  \item If the input nBDD is free (nFBDD),
    then the resulting circuit is
    \emph{decomposable} (DNNF). Further, if
    the input BDD is ordered (nOBDD), then the
    resulting circuit is \emph{structured}
    (SDNNF), with a \emph{linear vtree}
    corresponding to the OBDD order. Formally,
    a linear vtree corresponding to some
    variable order $X_1< \ldots < X_\ell$ is a
    vtree containing internal nodes
    $n_1,\ldots,n_{\ell-1}$ with $n_{i+1}$
    being a child of $n_i$ for $i\in
    \{1,\ldots,\ell-2\}$, with $X_i$ being a
    child of $n_i$ for $i\in
    \{1,\ldots,\ell-1\}$ and $X_\ell$ being a
    child of $n_{\ell-1}$.
    \item If the input nBDD  is \emph{unambiguous} (uBDD), then the resulting circuit is \emph{deterministic}. Further, if it is \emph{deterministic}  (BDD), then the resulting circuit is \emph{decision} (i.e., the only $\lor$-gates having more than one input in the circuit are the decision gates introduced in the transformation). The same holds in the case of free and structured diagrams: uFBDDs and uOBDDs respectively give d-DNNF and d-SDNNFs with a linear vtree, and FBDDs and OBDDs respectively give dec-DNNFs and dec-SDNNFs with a linear vtree.
    \item If the input BDD is a decision tree, then the result of the transformation is a Boolean formula (TODO check, depending on subclasses\mikael{what is there to check?})
\end{itemize}

Conversely, there is a correspondence from structured DNNF with a linear v-tree (TODO define), and OBDDs. TODO explain and note connection with decision/deterministic and nOBDD/uOBDD.

\subparagraph*{SDD} There are intermediate classes between determinism and decomposability:
\begin{itemize}
    \item Strong determinism: all or-gates have as inputs some ANDs, and the LHS of the ANDs are mutually exclusive
    \item Pre-SDD: Further, the disjunction of the LHSs is true (= the LHS partition the space)
\end{itemize}
These classes typically only make sense for strutured classes (indeed otherwise the first one is trivial). So an SDD is a structured pre-SDD.

\subparagraph*{Ordered circuits}
We also note that a class of circuits intermediate between DNNFs and SDNNFs has been proposed in~\cite{amarilli2017circuit}: this class requires that there is an order on the variables and that, for any $\land$-gate $g$, then $\var(g_1)$ and $\var(g_2)$ for its two inputs $g_1$ and $g_2$ are contained in disjoint intervals of the ordering. A structured circuit is always ordered, taking for order the traversal order on the leaves of a vtree, but the converse is not necessarily true.

Last, we note that there is a notion of $\land$-BDDs that has been proposed, extending BDDs with $\land$-gates. TODO, in particular for how order works there (see Florent?).

\paragraph*{Sharing.}
    When the underlying graph of a circuit is a tree (noting that each variable can occur multiple times), then we call the result a \emph{Boolean formula} (not to be confused with a Boolean function). If the circuit is in NNF, then the Boolean formula is in NNF. Examples of Boolean formulas in NNF include Boolean formulas in \emph{conjunctive normal form} (CNF), a conjunction of disjunction of literals; and Boolean formulas in \emph{disjunctive normal form} (DNF), a disjunction of conjunctions of literals. Note that Boolean formulas may still contain several occurrences of the same variable or literal; when we further impose that the circuit has a tree structure and that there is only one gate for each variable, we obtain what is called a \emph{read-once formula}.


\paragraph*{Smoothing.}
TODO

\paragraph*{Misc things.}
And-OBDDs?
