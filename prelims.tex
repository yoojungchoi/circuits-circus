We write sets of variables with uppercase boldface letters, such as $\X,\Y,\Z$, and single
variables with uppercase non-bold letters, such as~$X\in \X, Y\in \Y$, etc.

\paragraph*{Assignments and Boolean functions.}
Let~$\X$ be a finite set of variables.  An \emph{assignment of~$\X$} is a
function~$\ba: \X \to \{0,1\}$. We denote by~$\assign(\X)$ the set of all assignments
of~$\X$.  A \emph{Boolean function over~$\mathbf{X}$}
is a function~$f\colon \assign(\X)\to \{0,1\}$.
An assignment~$\ba$ of $\mathbf{X}$ is \emph{satisfying} if~$f(\ba) =1$ (also
denoted~$\ba \models f$).  We denote by~$\sat(f)\subseteq \assign(\X)$ the set of all
satisfying assignments of~$f$, and~$\ssat(f)$ the size of this set. 
%\yj{How about something like Models instead of SAT for the set of models?}


% \comm{Marcelo}{Notation $\x(X)$ has to be used to indicate the value of assignment $\x$ for variable $X$. This doesn't look very nice, could we use something like $\as$ for an assignment?}
% \comm{Antoine}{We move to $a$ and $\mathbf{a}$}
% M: doing this

\paragraph*{CNFs and DNFs.}
Let $\X$ a set of variables. A \emph{literal} is an expression of the form $X$
or $\lnot X$ for $X\in \X$. A \emph{clause} is a disjunction of literals, for
instance $X_1 \lor X_2 \lor \lnot X_3$. A \emph{formula in conjunctive normal
form}, or \emph{CNF} for short, is a formula that is a conjunction of clauses,
i.e., an expression of the form $\bigwedge_{i=1}^n C_i$ where each $C_i$ is a
clause. We call a conjunction of literals a \emph{term}. A \emph{formula in
disjunctive normal form}, or DNF, is a disjunction of terms i.e., an expression
of the form $\bigvee_{i=1}^n t_i$ where each $t_i$ is a term.
