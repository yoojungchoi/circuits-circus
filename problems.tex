\todo[inline]{YooJung \& Benjie}

We have seen that decision diagrams and logical circuits are compact representations of Boolean functions, and that circuits can also be used as a representation of polynomials and probability distributions. We now examine the common problems that one wants to solve; for simplicity, we focus on Boolean functions and later discuss the connections with other formalisms. In particular, we distinguish three kinds of problems, informally defined as follows:
\begin{itemize}
  \item \textbf{Queries}: A query maps one or more Boolean function to some simple output, e.g. Booleans or reals.
  %Boolean queries, and probabilistic variants: given a circuit and other
    %inputs, return a Boolean, or a probability, or a list of satisfying
   % assignments, etc.
  \item \textbf{Operations}: An operation maps one or more Boolean functions to another Boolean function.
  %given two or more circuits in the same formalism, represent the
    %result of an operation, as a circuit in the same class
  \item \textbf{Transformations}: A transformation maps a representation of a Boolean function into another representation of the same Boolean function.
  %given a circuit in class A, convert it to a circuit in
   % class B
\end{itemize}
An important question is the \emph{tractability} of these problems on different circuit representations, i.e. whether there exists a polynomial time algorithm for the query/operation/transformation. We will summarize known results on tractability in Sections \ref{sec:res_query}, \ref{sec:res_trans}, \ref{sec:res_ops}. 

\subsection{Queries}
% From Knowledge Compilation Map:
Consider a Boolean function $f$ over a set of variables $\X$. A \emph{query} $\query$ is a function that maps $f$, and possibly some other object(s), to some non-circuit object (e.g. Boolean, real). 

\begin{table}[]
\centering
\begin{tabular}{@{}lll@{}}
\toprule
\textbf{Query} & \textbf{Input}         & \textbf{Output} \\ \midrule
Consistency & None & $\sat(f) \neq \emptyset$  \\ \bottomrule
\end{tabular}
\label{tbl:bool_ queries}
\caption{Boolean Queries}
\end{table}


A query maps $f$ to an output as shown below:
\begin{itemize}
    \item Consistency: 1 if $\sat(f) \neq \emptyset$; 0 otherwise. \textit{(Satisfiability)}. 
    \item Consistency of partial assignment: given $f$ and a partial assignment, decide if there is a model extending the partial assignment.
    \item Validity: 1 if $\sat(f) = \assign(\X)$; 0 otherwise.
    \item Clausal entailment given a clause (disjunction of literals) $\gamma$: 1 if for every $\nu\in\sat(f)$, $\nu\models\gamma$; 0 otherwise. (Term entailment reduces to literal entailment, which is clause entailment.)
    \item Implicant given a term $\gamma$: 1 if $\sat(\gamma) \subseteq \sat(f)$; 0 otherwise. \yj{$\sat(\gamma)$ needs to be defined for variables $X$. Can alternatively define it in terms of assignments that model $\gamma$}
    \item (Weighted) model counting given a weight function $w:\assign(\X)\to\ZZ$: $\sum_{\nu\in\sat(f)} w(\nu)$
    \begin{itemize}
        \item Model counting (i.e., all weights are 1): $\ssat(f)$
        \item Often the weight of a model is defined as the product of weights of literals: $w(\nu)=\prod_{x\in X} w_{\nu(x)}$ \yj{Notation for literals using the valuation function?}
\item Can assume wlog that the weights are nonnegative integers, up to renormalizing. Note that this amounts to probability computation
    \end{itemize}
    \item Equivalence of functions $f_1$ and $f_2$: 1 if $\sat(f_1)=\sat(f_2)$; 0 otherwise. \textit{(Equality)}
    \item Sentential entailment of functions $f_1$ and $f_2$: 1 if $\sat(f_1)\subseteq\sat(f_2)$; 0 otherwise.
    \item Model enumeration
    \item PI (prime implicant) enumeration
    \item MAJSAT: at least half of the assignments satisfy
    % \item E-MAJSAT (given a subset of variables, is there an assignment such that, on the remaining of variables, at least half of the assignments satsify the formula), and max-sum (assuming we have a weight function, given a subset of variables, find the partial assignment of these variables maximizing the sum of the satisfying assignments of the entire circuit which are compatible with the partial assignment).
    \item Find the satisfying assignment of maximal weight / non-satisfying assignment of maximal weight. Caution: it's different from MAX-SAT on CNF (satisfying the maximal number of clauses). Note that a special case of this is ``find a minimal model''.
    \item Ranked enumeration of satisfying assignments
\end{itemize}

TODO Point out that we can always tractably test, given a circuit/diagram and a
valuation, if it is accepted.

\paragraph*{Probabilistic variants}
What if we relax tasks like ``equivaence of functions'' to be approximate, e.g., with probability $1-\delta$? TODO

\subsection{Operations}
\label{sec:operations}


\begin{table}[]
\centering
\begin{tabular}{@{}llll@{}}
\toprule
\textbf{Operation} & \textbf{Boolean}         & \textbf{Probabilistic} & \textbf{Database} \\ \midrule
$\formula | \varsubsetval$        & Conditioning             & Evidence/Instantiation & Selection         \\
$\exists \varsubset. \formula$                   & Forgetting (Existential Quantification)               & Marginalization        & Projection        \\
  $\formula_1 \wedge \formula_2$                     & Conjunction (bounded,
  i.e., only 2; or unbounded)             & Product                & Join              \\
  $\formula_1 \vee \formula_2$                  & Disjunction (bounded, i.e.,
  only 2; or unbounded)             & Sum/Max                & Union             \\
$\neg \formula_1$                   & Negation                 & ?                      & Negation          \\
$\forall \varsubset. \formula$                     & Universal Quantification & ?                      &   ?                \\ \bottomrule
\end{tabular}
\label{tbl:operations}
\caption{Operations}
\end{table}

We now define operations on Boolean functions. The key distinction between an operation and a query is that an operation maps Boolean function(s) to another Boolean function, as opposed to a scalar output.

\begin{definition}[Operation]
    A $n$-ary operation is a mapping $\op: \mathcal{F}^n \to \mathcal{F}$ where $\mathcal{F}$ is the set of Boolean functions over variables $\X$.
\end{definition}

\begin{definition}[Tractable Operation]
    A $n$-ary operation is tractable for input classes $\circuitclass_1, ... \circuitclass_n$ and output class $\circuitclass$ if there exists an algorithm (TODO: tractable algorithm?)
    that maps any $\circuit_1 \in \circuitclass_1, ..., \circuit_n \in \circuitclass_n$ to $\circuit \in \circuitclass$ such that $\circuit$ is logically equivalent to $\op(\circuit_1, ..., \circuit_n)$.

    TODO: change to have just one class
\end{definition}

In Table \ref{tbl:operations}, we provide a list of common operations on Boolean functions. It is worth noting that there are analogous operations on probabilistic circuits and on databases; we provide a translation in the Table. g

%Boolean  \textit{(Probabilistic)} \textbf{(DB)}
% \begin{itemize}
%     \item Conditioning \textit{(Evidence/Instantiation)} \textbf{(Selection)} (in particular setting a variable to a value; related to self-reducibility)
%     \item Forgetting \textit{(Marginalization)} \textbf{(Projection)}
%     \item Singleton forgetting (aka bounded forgetting): forget one variable
%     \item Conjunction \textit{(Product)} \textbf{(Join)} 
%     \item Bounded conjunction
%     \item Disjunction \textit{(Sum, max?)} \textbf{(Union)}
%     \item Bounded disjunction
%     \item Negation \textit{(? maybe $1-p$?)} \textbf{(Negation)}
%     \item Universal Quantification?
% \item Transform a circuit to the circuit of its minimal models
% \end{itemize}



\subsection{Transformations: Comparison between classes}

TODO? It's the special case of tractable operation when the 1-ary operation is the identity.

%\subsection{New content}

%1) Translation between classes (e.g. preserving properties you need). E.g. what is the "most general" input circuit class which guarantees an output class for an operation? 
% 2) Compositional queries?
% 3) Any automata-specific queries/operations
% MAX/MAP vs Counting: in decomposable Boolean circuits former is tractable, in decomposable arithmetic circuits latter is tractable.

