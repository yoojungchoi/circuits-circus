\todo{antoine}

In this section, we connect the notions of diagrams
(Section~\ref{sec:diagrams}) and circuits (Section~\ref{sec:circuits}). We do
so by explaining why diagrams are in fact specific classes of circuits, with a
direct translation from diagrams to circuits.
Furthermore, we show how the conditions that we have defined on diagrams can be
preserved by this transformation.

\paragraph*{Linear v-trees}
The translation from diagrams to circuits, when applied to ordered diagrams
(OBDDs), will produced structured circuits whose v-tree has a specific shape.
We define these v-trees, called \emph{right-linear v-trees}, and explain how
to obtain them from the order respected by the OBDD.

Formally, given an order $<$ on variables $\X$, we define the
\emph{right-linear vtree} $T_<$ obtained from~$<$ in the following way:
\begin{itemize}
  \item if
$\X$ consists of a single variable $x$ then $T_<$ is the singleton tree whose root
is a leaf node labeled by $x$;
\item if $\X$ consists of two variables $x<y$ then $T_<$ is the tree whose
  root is an internal node having $x$ as left child and $y$ as right child;
\item otherwise, letting $x$ be the smallest element of~$\X$ according to~$<$,
  we let $T_<$ be the tree whose root is an internal node having $x$ has
    left child and having as right child the root of the tree $T_{<'}$ obtained
    from the order $<'$ which is the restriction of~$<$ to $\X \setminus \{x\}$.
\end{itemize}

Note that the choice of right-linear v-trees is arbitrary; similar
construtions could be made with a left-linear v-tree.

\paragraph*{Converting diagrams to circuits}. 
We are now ready to state the immediate translation from diagrams to
circuits, and to explain how conditions on the circuit can be rephrased to
conditions on the diagram. This is a folklore result which has previously
appeared, e.g., Proposition 3.4 of~\cite{ACMS20}:

\begin{proposition}
  Given a nBDD $\D$, we can convert it in linear time to a circuit $C$
  that describes the same Boolean function.

  The translation satisfies the following variable structuredness
  conditions:
  \begin{itemize}
    \item If the input nBDD is free,
    then the resulting circuit is
    \emph{decomposable} (DNNF).
  \item If the input nBDD is ordered with an order $<$, then the
    resulting circuit is \emph{structured} with the right-linear vtree $T_<$
  \end{itemize}

  Cumulatively, the translation further satisfies the following level of ambiguity conditions:
  \begin{itemize}
    \item If the input nBDD is unambiguous then the resulting circuit
      is \emph{deterministic}.
    \item If the input nBDD is deterministic then the resulting circuit
      is \emph{decision}.
  \end{itemize}
  
  Last, if the input is a decision tree, then the output is a Boolean formula.
\end{proposition}

\begin{proof}
  We use the following general linear-time translation
from diagrams to circuits:
\begin{itemize}
  \item We replace true
leaves and false leaves respectively by
constant true and false gates, i.e.,
$\lor$-gates with no inputs and $\land$-gates
with no inputs respectively.
    \item We replace
internal nodes $n$ on a variable $X$ with a
gate defined like in the decision of decision
gates above; formally, let $g^1_0,\ldots, g^{k_0}_0$ (resp., $g^1_1,\ldots, g^{k_1}_1$)
be the translations of the nodes and
to which $n$ had edges labeled
with $0$ (resp., with $1$).
Construct a gate $g_0$ (resp., $g_1$) to be an $\lor$-gate of the gates $g^1_0,\ldots, g^{k_0}_0$ (resp., of $g^1_1,\ldots, g^{k_1}_1$).
We then translate
$n$ to a $\lor$-gate whose inputs are an
$\land$-gate conjoining $g_1$ and $X$,
and an $\land$-gate conjoining $g_0$
and a negation gate of~$X$.
    \item Last, the output (root) gate of the
circuit is the translation of the root of the
BDD.
\end{itemize}

One can check that the translation runs in
linear time and produces a circuit with the
same semantics as the BDD, i.e., a circuit that represents the same
  Boolean function; and that it has the stated properties. \antoine{TODO:
  more detail}
% 
% \begin{itemize}
%   \item If the input nBDD is free (nFBDD),
%     then the resulting circuit is
%     \emph{decomposable} (DNNF). Further, if
%     the input BDD is ordered (nOBDD), then the
%     resulting circuit is \emph{structured}
%     (SDNNF), with a \emph{linear vtree}
%     corresponding to the OBDD order. Formally,
%     a linear vtree corresponding to some
%     variable order $X_1< \ldots < X_\ell$ is a
%     vtree containing internal nodes
%     $n_1,\ldots,n_{\ell-1}$ with $n_{i+1}$
%     being a child of $n_i$ for $i\in
%     \{1,\ldots,\ell-2\}$, with $X_i$ being a
%     child of $n_i$ for $i\in
%     \{1,\ldots,\ell-1\}$ and $X_\ell$ being a
%     child of $n_{\ell-1}$.
%     \item If the input nBDD  is \emph{unambiguous} (uBDD), then the resulting circuit is \emph{deterministic}. Further, if it is \emph{deterministic}  (BDD), then the resulting circuit is \emph{decision} (i.e., the only $\lor$-gates having more than one input in the circuit are the decision gates introduced in the transformation). The same holds in the case of free and structured diagrams: uFBDDs and uOBDDs respectively give d-DNNF and d-SDNNFs with a linear vtree, and FBDDs and OBDDs respectively give dec-DNNFs and dec-SDNNFs with a linear vtree.
%     \item If the input BDD is a decision tree, then the result of the
%       transformation is a Boolean formula.
%       % (TODO check, depending on subclasses\mikael{what is there to check?})
% \end{itemize}
\end{proof}

\antoine{add an example}

